\documentclass[a4paper]{article}
\usepackage[utf8]{inputenc}
\usepackage[T1]{fontenc}
\usepackage[french]{babel}
\usepackage{geometry}
\geometry{hmargin=3.5cm,vmargin=3.5cm}
\usepackage{graphicx}
\usepackage{fancyhdr}
\usepackage{color}

\usepackage{listings}
\lstset{ %
  backgroundcolor=\color{white},   % choose the background color; you must add \usepackage{color} or \usepackage{xcolor}
  basicstyle=\small,               % the size of the fonts that are used for the code
  breakatwhitespace=false,         % sets if automatic breaks should only happen at whitespace
  breaklines=true,                 % sets automatic line breaking
  captionpos=b,                    % sets the caption-position to bottom
  commentstyle=\color{green},      % comment style
  deletekeywords={...},            % if you want to delete keywords from the given language
  escapeinside={\%*}{*)},          % if you want to add LaTeX within your code
  extendedchars=true,              % lets you use non-ASCII characters; for 8-bits encodings only, does not work with UTF-8
  frame=single,                    % adds a frame around the code
  keepspaces=true,                 % keeps spaces in text, useful for keeping indentation of code (possibly needs columns=flexible)
  keywordstyle=\color{blue},       % keyword style
  language=xml,                    % the language of the code
  morekeywords={Level,Cache,Architecture,...},            % if you want to add more keywords to the set
  numbers=left,                    % where to put the line-numbers; possible values are (none, left, right)
  numbersep=5pt,                   % how far the line-numbers are from the code
  numberstyle=\tiny\color{blue},  % the style that is used for the line-numbers
  rulecolor=\color{black},         % if not set, the frame-color may be changed on line-breaks within not-black text (e.g. comments (green here))
  showspaces=false,                % show spaces everywhere adding particular underscores; it overrides 'showstringspaces'
  showstringspaces=false,          % underline spaces within strings only
  showtabs=false,                  % show tabs within strings adding particular underscores
  stepnumber=2,                    % the step between two line-numbers. If it's 1, each line will be numbered
  stringstyle=\color{red},         % string literal style
  tabsize=2,                       % sets default tabsize to 2 spaces
  title=\lstname                   % show the filename of files included with \lstinputlisting; also try caption instead of title
}
\usepackage{hyperref}
\hypersetup{colorlinks=true}

\pagestyle{fancy}

\lhead{Rapport}
\rhead{Bibliothèque de Threads}
\lfoot{ENSEIRB-MATMECA}
\rfoot{Système d'Exploitation 2013-2014}

\begin{document}

\thispagestyle{empty}

\vspace{\stretch{1}}
\hrule
\begin{flushleft}
\Huge{Bibliothèque de Threads}\\
\end{flushleft}
\begin{flushright}
\huge\textbf{Rapport}\\
\end{flushright}
\hrule

\vspace{\stretch{1}}
\noindent\textbf{Auteurs :}
\emph{BOURDEAU Thibaud, DANDO Louis-Marie, HONORAT Alexandre, RINCEL Guillaume, SAGARDIA Elorri}\\
\\
\noindent\textbf{Encadrant :}
\emph{M. FAVERGE Mathieu}\\
\\
\noindent\textbf{Enseignant :}
\emph{M. GOGLIN Brice} 

\vspace{\stretch{1}}
\normalsize
\begin{center}
  Deuxième année, filière informatique\\
  Date : \today\\
  \textsc{Enseirb-Matmeca}
\end{center}


\newpage
%\tableofcontents

%\vspace{0.2\textheight}

\section*{Introduction}

\paragraph{}
Le projet de système d'exploitation consiste à développer une bibliothèque de \emph{threads}, basée sur l'interface de \texttt{p\_thread}. Dans un premier temps, celle-ci devra se contenter d'un seul thread noyau, ainsi tous les \emph{threads} que créera l'utilisateur seront exécutés de manière séquentielle. 

\paragraph{}
L'objectif minimal est de pouvoir implémenter toutes les fonctions de prototypes identiques à ceux de \texttt{p\_thread}, de sorte que tous les tests fournis foncionnent. Ce premier rapport présente l'avancement actuel du projet, en précisant les parties opérationnelles et non opérationnelles, ainsi que les choix effectués et à venir.

%\newpage
\section{Fonctionnement de la bibliothèque}
La bibliothèque, s'inspirant de \texttt{p\_thread}, implémente les fonctions suivantes:
\begin{itemize}
  \item \texttt{thread\_self}
  \item \texttt{thread\_create}
  \item \texttt{thread\_yield}
  \item \texttt{thread\_join}
  \item \texttt{thread\_exit}
\end{itemize}

\subsection{Organisation des sources}
\paragraph{}
L'implémentation de ces fonctions requiert la définition de certaines structures regroupées dans le répertoire \texttt{src/includes}. 
\begin{itemize}
\item La structure thread
\item La structure liste \\
Nous avons choisi la bibliothèque ccan\_list. Afin de pouvoir adapter le code à un autre type de liste, des fonctions d'abstraction ont été écrites. Celles-ci permettent d'effectuer les opérations basiques sur les listes : parcours, ajout, suppression.
\end{itemize}
\paragraph{}
Les programmes de tests fournis ont été rassemblés dans le répertoire \texttt{src/prog\_tests}.
\paragraph{}
Enfin, le fichier \texttt{thread.c} a été placé dans le répertoire \texttt{core.c} et contient le code principal du programme. 


\subsection{Choix effectués}
\paragraph{}
Il a été nécessaire de décider de la manière de stocker les threads en mémoire. C'est à ce niveau que la structure liste a été utilisée. Nous avons donc deux listes et un pointeur de thread.
\begin{itemize}
\item la waiting\_list
\item la ready\_list
\item le thread running
\end{itemize}
Actuellement, nous n'avons qu'un seul thread noyau, l'utilité d'une liste running ne nous a, par conséquent, pas semblé nécessaire.


\paragraph{}
Un choix s'est ensuite imposé sur la définition de la structure thread. \\
Celui-ci est de manière évidente, attaché à un contexte. Il doit donc être stocké dans la structure. \\
D'autre part, un thread, à chaque instant, est soit en attente, soit prêt à être executé soit en execution. C'est pourquoi, il est, respectivement, dans la waiting\_list, dans la ready\_list ou pointé par le thread running. Cette information sur le statut du thread est également stockée au sein de la structure thread elle-même pour des raisons d'optimisation. En effet, il serait inutilement coûteux de devoir parcourir chaque liste à la recherche d'un certain thread pour obtenir son statut. \\
Par ailleurs, au vu du contenu des fonctions de \texttt{thread.c}, nous avons choisi de marquer le thread\_main d'une manière différente. En effet, son allocation/desallocation s'effectue distinctement des autres. Ce marquage se traduit par un attribut \texttt{is\_main}. \\ 
Enfin, un attribut retval permet le stockage de la valeur renvoyée au niveau du thread\_join.



\section{Validation des résultats}

\subsection{Passage des tests fournis}
%% blabla sur le test 12 et le choix de la fuite mémoire
\subsection{Comparaison avec \texttt{p\_thread}}

\end{document}
