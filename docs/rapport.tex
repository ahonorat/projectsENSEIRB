\documentclass[a4paper]{article}
\usepackage[utf8]{inputenc}
\usepackage[T1]{fontenc}
\usepackage[french]{babel}
\usepackage{geometry}
\geometry{hmargin=3.5cm,vmargin=3.5cm}
\usepackage{graphicx}
\usepackage{fancyhdr}
\usepackage{color}

\usepackage{listings}
\lstset{ %
  backgroundcolor=\color{white},   % choose the background color; you must add \usepackage{color} or \usepackage{xcolor}
  basicstyle=\small,               % the size of the fonts that are used for the code
  breakatwhitespace=false,         % sets if automatic breaks should only happen at whitespace
  breaklines=true,                 % sets automatic line breaking
  captionpos=b,                    % sets the caption-position to bottom
  commentstyle=\color{green},      % comment style
  deletekeywords={...},            % if you want to delete keywords from the given language
  escapeinside={\%*}{*)},          % if you want to add LaTeX within your code
  extendedchars=true,              % lets you use non-ASCII characters; for 8-bits encodings only, does not work with UTF-8
  frame=single,                    % adds a frame around the code
  keepspaces=true,                 % keeps spaces in text, useful for keeping indentation of code (possibly needs columns=flexible)
  keywordstyle=\color{blue},       % keyword style
  language=xml,                    % the language of the code
  morekeywords={Level,Cache,Architecture,...},            % if you want to add more keywords to the set
  numbers=left,                    % where to put the line-numbers; possible values are (none, left, right)
  numbersep=5pt,                   % how far the line-numbers are from the code
  numberstyle=\tiny\color{blue},  % the style that is used for the line-numbers
  rulecolor=\color{black},         % if not set, the frame-color may be changed on line-breaks within not-black text (e.g. comments (green here))
  showspaces=false,                % show spaces everywhere adding particular underscores; it overrides 'showstringspaces'
  showstringspaces=false,          % underline spaces within strings only
  showtabs=false,                  % show tabs within strings adding particular underscores
  stepnumber=2,                    % the step between two line-numbers. If it's 1, each line will be numbered
  stringstyle=\color{red},         % string literal style
  tabsize=2,                       % sets default tabsize to 2 spaces
  title=\lstname                   % show the filename of files included with \lstinputlisting; also try caption instead of title
}
\usepackage{hyperref}
\hypersetup{colorlinks=true}

\pagestyle{fancy}

\lhead{Rapport}
\rhead{Bibliothèque de Threads}
\lfoot{ENSEIRB-MATMECA}
\rfoot{Système d'Exploitation 2013-2014}

\begin{document}

\thispagestyle{empty}

\vspace{\stretch{1}}
\hrule
\begin{flushleft}
\Huge{Bibliothèque de Threads}\\
\end{flushleft}
\begin{flushright}
\huge\textbf{Rapport}\\
\end{flushright}
\hrule

\vspace{\stretch{1}}
\noindent\textbf{Auteurs :}
\emph{BOURDEAU Thibaud, DANDO Louis-Marie, HONORAT Alexandre, RINCEL Guillaume, SAGARDIA Elorri}\\
\\
\noindent\textbf{Encadrant :}
\emph{M. FAVERGE Mathieu}\\
\\
\noindent\textbf{Enseignant :}
\emph{M. GOGLIN Brice} 

\vspace{\stretch{1}}
\normalsize
\begin{center}
  Deuxième année, filière informatique\\
  Date : \today\\
  \textsc{Enseirb-Matmeca}
\end{center}


\newpage
%\tableofcontents

%\vspace{0.2\textheight}

\section*{Introduction}

\paragraph{}
Le projet de système d'exploitation consiste à développer une bibliothèque de \emph{threads}, basée sur l'interface de \texttt{p\_thread}. Dans un premier temps, celle-ci devra se contenter d'un seul thread noyau, ainsi tous les \emph{threads} que créera l'utilisateur seront exécutés de manière séquentielle. 

\paragraph{}
L'objectif minimal est de pouvoir implémenter toutes les fonctions de prototypes identiques à ceux de \texttt{p\_thread}, de sorte que tous les tests fournis foncionnent. Ce premier rapport présente l'avancement actuel du projet, en précisant les parties opérationnelles et non opérationnelles, ainsi que les choix effectués et à venir.

%\newpage
\section{Fonctionnement de la bibliothèque}

 La bibliothèque, s'inspirant de \texttt{p\_thread}, implémente les
fonctions suivantes:
\begin{itemize}
  \item \texttt{thread\_self}
  \item \texttt{thread\_create}
  \item \texttt{thread\_yield}
  \item \texttt{thread\_join}
  \item \texttt{thread\_exit}
\end{itemize}

\subsection{Organisation des sources}

\paragraph{} L'implémentation de ces fonctions requiert la définition
de certaines structures regroupées dans le répertoire
\texttt{src/includes}. Les principales structures sont :
\begin{itemize}
\item La structure de thread.
\item La structure de liste.
\end{itemize} La bibliothèque \texttt{ccan\_list} a été choisie pour
représenter les listes. Mais afin de pouvoir adapter le code à un
autre type de liste, des fonctions d'abstraction ont été écrites dans
\texttt{src/others/manip\_list.c}. Celles-ci permettent d'effectuer
les opérations basiques sur les listes : parcours, ajout,
suppression. De même, aucun ordonnanceur -- autre que l'utilisation
basique des listes -- n'est implémenté pour l'instant, mais ses deux
fonctions principales (i.e. l'ajout d'un thread à la liste des threads
en attente ou prêts, ainsi que la récupération du prochain thread à
exécuter) sont regroupées dans un module à part :
\texttt{src/core/ordo.c}.
\paragraph{} Enfin les programmes de tests fournis ont été rassemblés
dans le répertoire \texttt{src/prog\_tests} et le fichier
\texttt{thread.c} a été placé dans le répertoire \texttt{src/core} et
contient le code principal du programme. La bibliothèque dynamique
créée est \texttt{thread.so}, ce qui permet de ne compiler qu'une
seule fois les tests.


\subsection{Choix effectués}

La programmation d'une bibliothèque de threads peut s'effectuer de
nombreuses façons, en effet différentes politiques d'ordonnancement,
de préemption, etc existent. La section suivante précisent ces choix.

\subsubsection{Stockage des threads}

\paragraph{} La manière de stocker les threads en mémoire est
déterminante sur l'ordonnancement. Les threads sont ainsi stockés
selon leur état dans une des trois structures suivantes :
\begin{itemize}
\item la \texttt{waiting\_list}, qui contient les threads ayant
terminé mais n'ayant pas été récupérés par leur parent ;
\item la \texttt{ready\_list}, qui contient la liste des threads prêts
à s'exécuter ;
\item le thread \texttt{running}, qui contient le threads en cours
d'exécution.
\end{itemize} Actuellement, un seul thread noyau est employé pour
exécuter l'ensemble des threads créés par l'utilisateur. L'utilité
d'une liste \texttt{running} ne nous a, par conséquent, pas semblé
nécessaire pour ce thread particulier. Par ailleurs notons que les
deux listes implémentées sont utilisées comme des FIFOs : un thread y
est toujours ajouté à la fin, et c'est le premier qui est récupéré.

\subsubsection{La structure thread}

\paragraph{} Premièrement, un choix s'est imposé sur la définition de
la structure thread. Un thread est de manière évidente, attaché à un
contexte et la structure doit donc le contenir.
\paragraph{} D'autre part, un thread, à chaque instant, est soit en
attente, soit prêt à être executé, soit en execution. Le statut du
thread est stocké au sein de la structure \texttt{thread} elle-même
pour des raisons d'optimisation. En effet, il serait inutilement
coûteux de devoir parcourir chaque liste à la recherche d'un certain
thread pour obtenir son statut, notamment dans le cas d'un
\texttt{join} qui doit vérifier si un thread est dans la
\texttt{waiting\_list}.
\paragraph{} Par ailleurs, au vu du contenu des fonctions de
\texttt{thread.c}, nous avons choisi de marquer le thread du
\texttt{main} d'une manière différente. En effet, son
allocation/desallocation s'effectue distinctement des autres : sa pile
n'est pas allouée par la bibliothèque. Ce marquage se traduit par un
attribut \texttt{is\_main}.
\paragraph{} Enfin, un attribut \texttt{retval} permet le stockage de
la valeur renvoyée au niveau du \texttt{thread\_join}. Pour gérer le
cas d'un thread qui n'appelle pas la fonction \texttt{thread\_exit},
une fonction auxiliaire a été implémentée dans la bibliothèque et
prend comme argument la fonction passée à \texttt{thread\_create}
ainsi que son argument d'appel. Il s'agit d'un \emph{wrapper} qui lui
appelle la fonction \texttt{thread\_exit} dans tous les cas.

\subsubsection{Initialisation et Terminaison}

\paragraph{}
Les fonctions de la bibliothèque nécessitent que
certains objets soient instanciés au préalable. C'est le cas des deux
listes initialisées vides et du thread running qui est initialisé avec le thread du
\texttt{main}. De la même façon, avant de quitter le programme, le
thread running doit être détruit (les listes ready et waiting sont
déjà vides à ce moment là). C'est pourquoi les
fonctions \texttt{thread\_init} et \texttt{thread\_quit} sont
appelées automatiquement car définies avec des attributs
\emph{constructor} et \emph{destructor} respectivement.

\paragraph{}
D'autre part, afin d'éviter la duplication de code, une
fonction d'allocation \texttt{thread\_construct} et de désallocation
\texttt{thread\_destruct} internes au fichier \texttt{thread.c} ont
été ajoutées. La fonction d'allocation est appelée à la fois par
\texttt{thread\_init} et par \texttt{thread\_create} et c'est elle qui
effectuera des traitements différents en fonction du thread (running
et/ou main ou cas général). De la même manière, la fonction de
désallocation peut être appelée par \texttt{thread\_quit} ou
\texttt{thread\_join}. En effet, dès qu'un thread est récupéré par son
parent, il doit être détruit.

\subsubsection{Création de contexte}

\paragraph{}
Le choix entre \texttt{swapcontext} et \texttt{setcontext} dépend du
 besoin de retenir le contexte précédent ou non. C'est ainsi que 
\texttt{thread\_yield} utilise \texttt{swapcontext} alors que 
\texttt{run\_thread}, appelé uniquement lorsque la \texttt{ready\_list}
ne contient qu'un élément, utilise \texttt{setcontext}. 

\subsubsection{Le thread issu de main}

\paragraph{}
Le thread principal est le seul thread que la bibliothèque ne crée
pas : son traitement demande par conséquent quelques adaptations.
En particulier, sa pile n'est pas allouée. Il est donc hors de
question de la libérer - du reste, nous avons été incapables de
récupérer son adresse.

\paragraph{}
Par ailleurs, la fonction \texttt{exit} est la seule à pouvoir libérer proprement la mémoire
allouée pour le thread principal. Cependant, si elle est exécutée depuis un autre contexte que le thread principal
(avec sa propre pile), la fonction \texttt{exit} dysfoncionne.
Il faut donc recourir à une astuce pour, au moment où le thread \texttt{main}
appelle \texttt{thread\_exit}, sauvegarder un contexte sur le point
d'apppeler \texttt{exit} tout en le remplaçant par celui d'un thread
prêt.

\paragraph{}
Au moment de quitter le programme (si lors d'un appel à
\texttt{thread\_exit}, il n'y a plus de thread prêt à qui donner la
main -- cf test 12-join), on rend la main à ce contexte, dont la pile
est la pile principale, et qui exécute donc \texttt{exit} correctement.


\section{Validation des résultats}

La validation des résultats consiste principalement en une comparaison entre le comportement de le bibliothèque implémentée lors du projet et celui de \texttt{p\_thread}. Il convient à la fois de vérifier la correspondance des résultats, mais aussi d'expliquer les différences de temps de calcul entre les deux bibliothèques pour des tests choisis. 

\subsection{Scripts de tests}

\paragraph{}
Trois scripts permettent de tester les tests fournis, ainsi que ceux implémentés dans le cadre du projet. 

\paragraph{\texttt{run\_tests.sh}}
Ce script lance tous les tests et effectue un \texttt{diff} entre les sorties des binaires utilisant \texttt{p\_thread} et celles des binaires utilisant la bibliothèque du projet. Les \texttt{diff} ne devraient rien afficher en dehors des différences dues au temps d'exécution ou aux adresses.

\paragraph{\texttt{valgrind\_tests.sh}}
Le principe est identique au script précédent, il ds'agit cette fois de comparer les sorties de valgrind.

\paragraph{\texttt{bench/create\_graph.sh}}
Les tests fournissant leur temps d'exécution en sortie sont ici exécutés afin de comparer les performances des deux bibliothèques. Cinq graphiques sont automatiquement générés, en faisant une moyenne sur dix lancements. Cela permet de s'assurer de la pertinence des performances sur des plateformes différentes.

\subsection{Passage des tests}

\paragraph{Tests fournis}
Tous les tests fournis par M. Goglin fonctionnent avec la bibliothèque implémentée \texttt{thread.so}. Aucune erreur mémoire ne se présente pour ces tests (aucune fuite mémoire, aucune erreur de lecture). \textsf{Valgrind} détecte cependant un changement de pile (et émet un avertissement). Mais cette manipulation est maîtrisée : elle permet en effet de désallouer correctement la pile dans le test \texttt{12-join-main}.

\paragraph{Tests implémentés/modifiés}
La logique des tests implémentés dans le cadre du projet était de vérifier l'exécution des nouvelles fonctionnalités. Ceux-ci ont déjà été présentés dans les sections de ces nouvelles fonctionnalités, notons simplement que leur exécution ne provoque pas non plus d'erreur ni de fuite mémoire.

\paragraph{Rapidité de la bibliothèque du projet}
Les graphiques générés automatiquement par le script dans \texttt{bench/} permet de s'assurer qu'en toute circonstance, la bibliothèque du projet permet une exécution plus rapide des tests. Cela est toutefois a relativiser avec le fait que les tests ne sont pas prévus pour effectuer des opérations parallèles, où seuls une implémentation de threads noyaux aurait pu permettre à la bibliothèque d'être peut-être aussi performante.

\end{document}
