\section{Fonctionnement de la bibliothèque -- version minimale}

 La bibliothèque, s'inspirant de \texttt{p\_thread}, implémente les
fonctions suivantes:
\begin{itemize}
  \item \texttt{thread\_self}
  \item \texttt{thread\_create}
  \item \texttt{thread\_yield}
  \item \texttt{thread\_join}
  \item \texttt{thread\_exit}
\end{itemize}
La plupart des arguments sont similaires à quelques exceptions près : dans la version du projet il n'y a pas d'attribut de thread à passser à \texttt{thread\_create} et la fonction que le \emph{thread} exécute ne peut avoir qu'un seul argument \texttt{void *}.

\subsection{Organisation des sources}

\paragraph{} 
L'implémentation de ces fonctions requiert l'utilisation d'un certain nombre de fonctions annexes, non visibles par l'utilisateur. Par conséquent les sources sont réparties dans les dossiers \texttt{includes}, \texttt{core} et \texttt{others}.

\paragraph{}
En particulier, la bibliothèque \texttt{ccan\_list} a été choisie pour
représenter les listes et afin de pouvoir adapter le code à un
autre type de liste, des fonctions d'abstraction ont été écrites dans
\texttt{src/others/manip\_list.c}. Celles-ci permettent d'effectuer
les opérations basiques sur les listes : parcours, ajout,
suppression. 

\paragraph{}
De même, les deux fonctions principales de l'ordonanceur 
(i.e. l'ajout d'un thread à la liste des threads
en attente ou prêts, ainsi que la récupération du prochain thread à
exécuter) sont regroupées dans un module à part :
\texttt{src/core/ordo.c}. La version minimale ordonnance avec une politique FIFO : 
un nouveau thread est toujours ajouté en fin de liste et le prochain à exécuter est le premier de la liste.

\paragraph{}
Enfin les programmes de tests fournis ont été rassemblés
dans le répertoire \texttt{src/prog\_tests} et le fichier
\texttt{thread.c} a été placé dans le répertoire \texttt{src/core} et
contient le code principal du programme. La bibliothèque dynamique
créée est \texttt{thread.so}, ce qui permet de ne compiler qu'une
seule fois les tests.


\subsection{Choix effectués}

Une bibliothèque de threads peut s'utiliser de
nombreuses façons, en effet différentes politiques d'ordonnancement,
de préemption, etc. existent. La section suivante précise les choix faits 
pour la version minimale.

\subsubsection{Stockage des threads}

\paragraph{} La manière de stocker les threads en mémoire est
déterminante sur l'ordonnancement. Les threads sont ainsi stockés
selon leur état dans une des trois structures suivantes :
\begin{itemize}
\item la \texttt{waiting\_list}, qui contient les threads ayant
terminé mais n'ayant pas été récupérés par leur parent ;
\item la \texttt{ready\_list}, qui contient la liste des threads prêts
à s'exécuter ;
\item le thread \texttt{running}, qui contient le thread en cours
d'exécution.
\end{itemize} Actuellement, un seul thread noyau est employé pour
exécuter l'ensemble des threads créés par l'utilisateur. L'utilité
d'une liste \texttt{running} ne nous a, par conséquent, pas semblé
nécessaire pour ce thread particulier. Par ailleurs notons que les
deux listes implémentées sont utilisées comme des FIFOs : un thread y
est toujours ajouté à la fin, et c'est le premier qui est récupéré.

\subsubsection{La structure thread}

\paragraph{} 
Premièrement, un choix s'est imposé sur la définition de
la structure thread. Un thread est de manière évidente, attaché à un
contexte et la structure doit donc le contenir.

\paragraph{} 
D'autre part, un thread, à chaque instant, est soit terminé, soit en
attente d'un fils, soit prêt à être exécuté, soit en exécution. Le statut du
thread est stocké au sein de la structure \texttt{thread} elle-même
pour des raisons d'optimisation. En effet, il serait inutilement
coûteux de devoir parcourir chaque liste à la recherche d'un certain
thread pour obtenir son statut, notamment dans le cas d'un
\texttt{join} qui doit vérifier si un thread est dans la
\texttt{waiting\_list}.

\paragraph{} 
Par ailleurs, au vu du contenu des fonctions de
\texttt{thread.c}, nous avons choisi de marquer le thread du
\texttt{main} d'une manière différente. En effet, son
allocation/désallocation s'effectue distinctement des autres : sa pile
n'est pas allouée par la bibliothèque. Ce marquage se traduit par un
attribut \texttt{is\_main}.

\paragraph{}
Enfin, un attribut \texttt{retval} permet le stockage de
la valeur renvoyée au niveau du \texttt{thread\_join}. Pour gérer le
cas d'un thread qui n'appelle pas la fonction \texttt{thread\_exit},
une fonction auxiliaire a été implémentée dans la bibliothèque et
prend comme argument la fonction passée à \texttt{thread\_create}
ainsi que son argument d'appel. Il s'agit d'un \emph{wrapper} qui lui
appelle la fonction \texttt{thread\_exit} dans tous les cas.

\subsubsection{Initialisation et Terminaison}

\paragraph{}
Les fonctions de la bibliothèque nécessitent que
certains objets soient instanciés au préalable. C'est le cas des deux
listes initialisées vides et du thread running qui est initialisé avec le thread du
\texttt{main}. De la même façon, avant de quitter le programme, le
thread running doit être détruit (les listes ready et waiting sont
déjà vides à ce moment là). C'est pourquoi les
fonctions \texttt{thread\_init} et \texttt{thread\_quit} sont
appelées automatiquement car définies avec des attributs
\emph{constructor} et \emph{destructor} respectivement.

\paragraph{}
D'autre part, afin d'éviter la duplication de code, une
fonction d'allocation \texttt{thread\_construct} et de désallocation
\texttt{thread\_destruct} internes au fichier \texttt{thread.c} ont
été ajoutées. La fonction d'allocation est appelée à la fois par
\texttt{thread\_init} et par \texttt{thread\_create} et c'est elle qui
effectuera des traitements différents en fonction du thread (running
et/ou main ou cas général). De la même manière, la fonction de
désallocation peut être appelée par \texttt{thread\_quit} ou
\texttt{thread\_join}. En effet, dès qu'un thread est récupéré par son
parent, il doit être détruit.

\subsubsection{Création de contexte}

\paragraph{}
Le choix entre \texttt{swapcontext} et \texttt{setcontext} dépend du
 besoin de retenir le contexte précédent ou non. C'est ainsi que 
\texttt{thread\_yield} utilise \texttt{swapcontext} alors que 
\texttt{run\_thread}, appelé uniquement lorsque la \texttt{ready\_list}
ne contient qu'un élément, utilise \texttt{setcontext}. 

\subsubsection{Le thread issu de main}

\paragraph{}
Le thread principal est le seul thread que la bibliothèque ne crée
pas : son traitement demande par conséquent quelques adaptations.
En particulier, sa pile n'est pas allouée. Il est donc hors de
question de la libérer - du reste, nous avons été incapables de
récupérer son adresse.

\paragraph{}
Par ailleurs, la fonction \texttt{exit} est la seule à pouvoir libérer proprement la mémoire allouée pour le thread principal. Cependant, si elle est exécutée depuis un autre contexte que le thread principal (avec sa propre pile), la fonction \texttt{exit} dysfonctionne.%ol : explications?
Il faut donc recourir à une astuce pour, au moment où le thread \texttt{main}
appelle \texttt{thread\_exit}, sauvegarder un contexte sur le point
d'appeler \texttt{exit} tout en le remplaçant par celui d'un thread
prêt.

\paragraph{}
Au moment de quitter le programme (si lors d'un appel à
\texttt{thread\_exit}, il n'y a plus de thread prêt à qui donner la
main -- cf test 12-join), on rend la main à ce contexte, dont la pile
est la pile principale, et qui exécute donc \texttt{exit} correctement.
